\documentclass[11pt]{article}
\usepackage{graphicx} % Required for inserting images
\usepackage[utf8]{inputenc}
\usepackage{amsmath, amssymb}
\usepackage{mathtools}
\usepackage{physics}
\usepackage{braket}
\usepackage{geometry}
\usepackage{setspace}
\usepackage{sectsty}
\usepackage{titling}
\usepackage{tocloft}
\usepackage{hyperref}
\geometry{a4paper,  margin=1in}
\setstretch{1.5}
\numberwithin{equation}{section}
\AtBeginDocument{\Large \selectfont}
\title{Quantum states and mode decomposition (Notes)}
\author{YILUN LIN}
\date{June 2025}

\begin{document}
\maketitle
\Large\textbf{Overview}
\newline
{\Large These notes summarize how quantum states in quantum field theory are represented in Fock space, how field operators are mode-expanded, and how arbitrary states are expanded in the momentum eigenbasis.}

\section{Fock Space and Quantum States}
\textbf{Core idea:} Fock space organizes states with varying particle number. \\
\begin{itemize}
 \item\textbf{Vacuum state}  $\lvert0\rangle$: defined by
 \begin{equation}
 a_{\mathbf p}\, \lvert0\rangle = 0,\quad \forall\, \mathbf p.
 \end{equation}
 \textit{Physical meaning:} The vacuum contains no particles in any momentum mode.
 \item \textbf{$n$-particle momentum eigenstate}:
 \begin{equation}
 \lvert \mathbf p_1, \dots,\mathbf p_n\rangle = a_{\mathbf p_1}^\dagger\cdots 
 a_{\mathbf p_n}^\dagger\,\lvert0\rangle,
 \end{equation}
 satisfying
 \begin{equation}
 \hat P^\mu\, \lvert \mathbf p_1, \dots,\mathbf p_n\rangle = \Bigl(\sum_i p_i^\mu\Bigr)\,\lvert \mathbf p_1,\dots,\mathbf p_n\rangle.
 \end{equation}
 \textit{Physical meaning:} Each excitation adds a particle with momentum $p_i$,
 total four-momentum is sum of individual momenta.
\\[1em]
\end{itemize}

\section{Field Operator and Mode Decomposition}
\textbf{Core idea:} The field operator creates or annihilates quanta at spacetime points, expanded in momentum modes. \\[1em]
For a free real scalar field $\hat\phi(x)$ in the Heisenberg picture:
\begin{equation}
   \hat\phi(t,\mathbf x) = \int\!\frac{d^3p}{(2\pi)^3}\, \frac{1}{\sqrt{2E_{\mathbf p}}}\Bigl[a_{\mathbf p} e^{-i(E_{\mathbf p}t-\mathbf p\cdot\mathbf x)} + a_{\mathbf p}^\dagger e^{+i(E_{\mathbf p}t-\mathbf p \cdot \mathbf x)}\Bigr]. 
\end{equation}   
\begin{equation}
   [a_{\mathbf p}, \, a_{\mathbf p'}^\dagger] = (2\pi)^3 \delta^{(3)} (\mathbf p-\mathbf p').
\end{equation}
\\[1em]
\textit{Physical meaning:} Each Fourier mode behaves as an independent harmonic oscillator; the field operator sums over these to create or annihilate particles at $(t,\mathbf x)$.
\\[1em]
\section{Expansion of an Arbitrary State}
\textbf{Core idea:} Any state in Fock space can be viewed as a superposition of 
$n$-particle momentum eigenstates. \\
\begin{equation}
  \lvert\Psi\rangle = \sum_{n=0}^\infty \frac{1}{n!} \int\!\prod_{i=1}^n\frac{d^3p_i}{(2\pi)^3}\; \Psi_n(\mathbf p_1, \dots, \mathbf p_n)\; a^\dagger_{\mathbf p_1}\cdots a^\dagger_{\mathbf p_n}\,\lvert0\rangle.
\end{equation}
\textbf{Coefficient functions:}
\begin{equation}
  \Psi_n(\mathbf p_1, \dots, \mathbf p_n) = \langle0\mid a_{\mathbf p_n}\cdots
  a_{\mathbf p_n}\mid\Psi\rangle.
\end{equation}
\textit{Physical significance:} $\Psi_n(\{\mathbf p_i\})$ is the $n$-particle wavefunction in momentum space, encoding the amplitude for finding particles with those momenta.
\\[1em]

\section{Computing the Coefficient Functions}
\textbf{Core idea:} Projecting the state onto basis states extracts the coefficient wavefunctions. \\[1em]
\textbf{Example:  One-Particle States}
\begin{equation}
   \lvert\Psi\rangle = \int \! \frac{d^3p'}{(2\pi)^3}\; f(\mathbf p')\; 
   a_{\mathbf p'}^\dagger\lvert0\rangle.
\end{equation}
Project:
\begin{equation}
   \Psi_1(\mathbf p) 
   = \langle0\mid a_{\mathbf p}\, \lvert\Psi\rangle 
   = \int \! \frac{d^3p'}{(2\pi)^3}\; f(\mathbf p')\; \langle0\mid a_{\mathbf p} 
     a_{\mathbf p'}^\dagger \mid0\rangle 
   = f(\mathbf p).
\end{equation}
\textit{Physical significance:} The function $f(\mathbf p)$ chosen in the state construction directly becomes the momentum-space wavefunction.
\\[1em]

\section{Position-Space Wavefunction}
\textbf{Core idea:} Field operators can create localized excitations, giving position-space amplitudes. \\[1em]
\textbf{Definition}
\begin{equation}
   \hat\psi^\dagger(\mathbf x) = \int \! \frac{d^3p}{(2\pi)^3\sqrt{2E_{\mathbf p}}}\, a_{\mathbf p}^\dagger e^{-i\mathbf p\cdot\mathbf x}.
\end{equation}
\textbf{Position eigenstate:}
\begin{equation}
   \lvert\mathbf x\rangle = \hat\psi^\dagger(\mathbf x)\lvert0\rangle,
   \quad\langle\mathbf x \mid\mathbf y\rangle = \delta^{(3)}(\mathbf x-\mathbf y).
\end{equation}
\textbf{$N$-particle position wavefunction:}
\begin{equation}
   \Psi(\mathbf x_1, \dots, \mathbf x_n) = \langle0\mid\hat\psi(\mathbf x_1)
   \cdots\hat\psi(\mathbf x_n)\mid\Psi\rangle.
\end{equation}
\textbf{Physical meaning:} $\Psi(\{\mathbf x_i\})$ encodes the joint probability 
amplitude for finding particles at spatial points.
\\[1em]

\section{Practical Steps}
\textbf{Core idea:} Workflow for translating abstract states to usable wavefunctions.   \\
\begin{enumerate}
   \item Choose a basis (momentum, angular momentum, position, or other physical quantities) according to symmetry or measurement.
   \item Apply corresponding annihilation or field operators to project $\lvert\Psi\rangle$.
   \item Extract coefficient functions (wavefunctions or wavefunctionals).
   \item Use these in integrals to compute physical observables or correlation functions.
\end{enumerate}

\end{document}
